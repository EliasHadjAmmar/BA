\documentclass{article}
\usepackage[utf8]{inputenc}
\usepackage[a4paper,left=2.5cm,right=2.5cm,top=2.5cm,bottom=2.5cm]{geometry} 

\usepackage{natbib} % we need this so we can use citation and bib properly
\usepackage{enumitem} % allow you to customize your items (like margins etc.)
\usepackage{xurl} % allow URL breaks at any alphanumerical character
\usepackage{hyperref} % allows you to add hyperlink
\usepackage{amsmath} % allows you to use most mathematical features
\usepackage{float,fancyhdr}
\usepackage{amssymb}
\usepackage{setspace} % allows you to change the line spacing
\usepackage{xcolor}
\usepackage{graphicx} % we need this so we can add figures

\title{Does Political Integration Make Cities Grow Faster?\\
Evidence From Dynastic Takeovers\footnote{I thank Davide Cantoni, Matthias Weigand, and my RA colleagues at the Chair of Economic History for helpful comments and feedback.}
}
\author{Elias Hadj Ammar}
\date{January 2023}

\begin{document}
\onehalfspacing
\maketitle
\thispagestyle{empty}

\begin{abstract}
I use historical data on construction activity and territorial history to estimate the effect of state size on city growth in the Holy Roman Empire. I exploit variation from lineage extinctions, which caused cities to become part of different lineages' territorial holdings. I find that an X\% increase in territory size is associated with a Y\% increase in construction activity.
\end{abstract}




\newpage

\setcounter{page}{1}
\doublespacing
 
\section{Introduction}

The political history of Europe, until fairly recently, was defined by ruling families passing their territorial holdings down in hereditary monarchies. Land and cities changed hands often throughout the history of Europe. Which territory a city belonged to had economic as well as political consequences, and it is not hard to imagine that the fortunes of whole stretches of land could be decided by such dynastic takeovers. For example, (good example of a takeover and its consequences here).

In this paper I investigate changes in territory size as one possible channel through which dynastic takeovers could affect development. There exists a literature on the optimal size of state (e.g. \cite{as1997}, \cite{aw1998}, \cite{asw2000}, \cite{easterly2000}). This literature focuses on the tradeoffs between scale effects in government on one hand and conflict on the other, as well as trade and openness.

background on the hre / motivation for using city-level data and territory switches

I use a stacked event study framework with varying intensity of treatment. The treatment is the size difference between the old and the new rulers' territorial holdings. I will argue that the takeover constitutes a shock to territory size that is exogenous to development. The event study shows that cities with different takeover experiences did not differ in growth (conditional on city and year fixed effects) before their respective takeovers. 

The outcome variable used throughout the analysis is yearly construction activity. I go to some lenghts to show that this is a good proxy for growth, comparing the present data from the Deutsches St\"adtebuch to the craftsmen real wage time series by Allen for available cities. The two measures generally align.

Results indicate that territory size has a large, significant and positive effect on city growth. Cities that joined a twice-as-big territory experienced 10 \% higher construction activity in subsequent years. Splitting the sample into cities from big, medium-sized and small territories reveals that the effect is mostly driven by cities from small territories. However, cities in big territories generally changed hands less often, making it more difficult to find a meaningful and significant effect in that subsample.

The literature says that benefits of larger states include free trade within a larger area. Trade within the HRE was not exactly free. I'm not currently aware of data on historical trade flows between German cities. However, I use the establishment of historical markets (Cantoni et al) and the presence of Jews (data from Voigtl\"ander and Voth) to investigate this channel. I find that trade increases (cities are (insignificantly) more likely to establish markets, and the presence of Jews increases) when cities join larger territories.

\section{Literature review}

I will group literature thematically. Strands of relevant literature: 

\subsection{State size and capacity}

Theory: \cite{as1997}

Historical evidence: \cite{cervellati2022}, \cite{koyama2017}, \cite{schoenholzer2018}, \cite{sng2014}

\subsection{Economic and political integration}

Theory: \cite{sapir2011}, \cite{asw2000}

Historical evidence: \cite{chilosi2013}, \cite{kindleberger1975}, \cite{campos2019}, \cite{wolf2009}, \cite{chen2004}

\subsection{HNEs in the HRE}

Papers using similar approaches: \cite{cantoni2011}

\subsection{Hereditary rule as a HNE}

Theory: \cite{besley2017}

Evidence: \cite{olken2005}, \cite{dube2020}, \cite{ottinger2022}


\section{Cities, lineages, and territories}

\section{Empirical approach}

\subsection{Pooled OLS regression}

\subsection{Obtaining a dataset of stacked natural experiments}

\subsection{Stacked event study and extinction IV}

\section{Results}

\section{Validity and robustness}

\section{Conclusion}



\newpage
\onehalfspacing
\bibliographystyle{chicago}
\bibliography{references.bib}

\newpage
\section*{Tables and figures}


\end{document}
