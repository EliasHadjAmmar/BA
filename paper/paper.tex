\documentclass[11pt, a4paper]{article}
\usepackage[utf8]{inputenc}
\usepackage[left=3cm,right=3cm,top=2.5cm,bottom=2.5cm]{geometry}  
% set 11pt and right=6cm for submission

\usepackage{natbib} % we need this so we can use citation and bib properly
\usepackage{enumitem} % allow you to customize your items (like margins etc.)
\usepackage{xurl} % allow URL breaks at any alphanumerical character
\usepackage{hyperref} % allows you to add hyperlink
\usepackage{amsmath} % allows you to use most mathematical features
\usepackage{float,fancyhdr}
\usepackage{amssymb}
\usepackage{setspace} % allows you to change the line spacing
\usepackage{xcolor}
\usepackage{graphicx} % we need this so we can add figures
\usepackage{booktabs}

\newcommand{\decoline}{% pretty "scene change" macro
  \par
  \vspace{0.33\baselineskip}
  \hfill
  {\large* * *}
  \hfill
  \vspace{0.33\baselineskip}
\par}


\title{Conflict Driven Evolution and Urban Development: \\ Evidence from the Holy Roman Empire\footnote{
I thank Lara Andermann, Leonie Bielefeld, Daniel Bursian, Davide Cantoni, Emilio Esguerra, Filip Milojevic, Johannes Vorbach and Matthias Weigand for helpful comments and feedback.
}
}
\author{Elias Hadj Ammar}
\date{June 27, 2023}

\begin{document}

\begin{center}\uppercase{Ludwig-Maximilians-Universität München}\end{center}
\begin{center}\uppercase{Seminar für Wirtschaftsgeschichte}\end{center}

\vspace{3cm}


\begin{center}
    {\centering\huge{{Conflict-Driven Evolution and Urban Development:\\
Evidence from the Holy Roman Empire
}}}
\end{center}
\date{\vspace{-5ex}}
\thispagestyle{empty}


%%%Block Masterarbeit:
\begin{center}
\begin{large}
\begin{Large}
Bachelorarbeit\\
\end{Large}
im Studiengang 'Volkswirtschaftslehre' \\
\end{large}
\end{center}
\begin{center}
vorgelegt von\\
\begin{large}
Elias Hadj Ammar\\
\end{large}
\end{center}
\vspace{1cm}
\begin{center}
Betreuerin:\\
\begin{large}
Leonie Bielefeld\\
\end{large}
\end{center}
%%%Ende Block Masterarbeit

\begin{center}
\begin{large}
München, den 27. Juni 2023 \\
\end{large}
\end{center}

\vspace{1,5cm}

\begin{center}
\url{Elias.Ammar@campus.lmu.de}\\
Matrikelnr.: 12113698 \\
\end{center}

\clearpage


\onehalfspacing
\maketitle
\thispagestyle{empty}

\begin{abstract}
How does switching between states affect the long-run development of cities, and why? I study this question in the setting of the Holy Roman Empire, using historical data on construction activity and the territorial history of cities. I specifically test the hypothesis that cities benefitted from ownership changes because competition between states selected for higher-quality government. While I show that cities grow faster in the long run after switching states, the evidence that conflict-driven selection played a role in this is less compelling.
\end{abstract}

\newpage

\setcounter{page}{1}
\doublespacing


% ######################################################

\section{Introduction}



The European state system was forged in conflict. No other continent at any time experienced more war than Europe in the early modern era (\citealp{voigtlnder2013}, p. 174). Wars bring death and devastation to the societies affected by them, but their effects also manifest on the political map --- they play out in a diverse ecosystem of polities that come and go and change size over time. Borders are redrawn and territory changes hands. One can indeed imagine this as an evolutionary process: states compete for resources and powerful states eventually absorb weaker ones. As a result, traits that increase state power --- the ability to win conflicts against rivals --- get selected for, while unfavourable traits are doomed to disappear \citep{levine2021}.

In this paper I study the role of this process in the economic development of the Holy Roman Empire. In particular, I estimate the effect of state power, the trait that conflict selects for, on urban development. If the effect is positive, then the conflict driven evolution of states may partially explain the eventual ascendancy of Europe: by selecting for military success, Europe's violent history may have accidentally produced fiscal capacity, efficient legal systems, and prospering middle classes.

I build my paper on the theory of conflict driven evolution, outlined above, developed by \cite{levine2013, levine2021, levine2022}. Its main contribution is to explain why hegemonies arose in East Asia while balances of power could prevail in Europe: unlike China, the major powers of continental Europe were frequently preyed upon by strong outsiders --- the Vikings, the Swedes, and the English --- which prevented any one state from growing too strong (\citealp{levine2021}, p. 439). It is worth highlighting how the authors model war between states: you win them with state power, and state power is simply aggregate production in excess of the subsistence level. If the most successful states are the ones with the biggest surplus, selection should have made Europe increasingly prosperous.\footnote
{
The basic idea of Darwinian selection has previously been borrowed by economists - for example by \cite{galor2002} who argue that humans are subject to selection and accumulate (economically) favourable traits over the course of all history. More infamously, \cite{clark2007} employs it in an attempt to explain why the Industrial Revolution took place in Europe, arguing that at some point the European rich began to out-breed the poor and passed their middle-class values on to their offspring.
}

The idea that war and conflict can raise the standard of living dates back to \cite{malthus1798}. He reasoned that "positive checks" on the population (war, disease, famine) lead to higher land-labour ratios and therefore higher levels of income per capita. \cite{voigtlnder2013} build upon this Malthusian mechanism and add an explanation for the incessant wars: taxes on those higher incomes filled the treasuries of the rulers, which increased their demand for war --- a luxury good ---, which in turn kept population low. 

By contrast, my story is not about population dynamics but institutions (although it is compatible with Malthusianism). A longstanding tradition in economics emphasises the role of institutional factors in long-run growth (e.g. \cite{north1970}, \cite{delong1993}, \cite{ajr2001}). From this vast literature one theme is especially relevant here: the institution of fiscal capacity, or the state's ability to collect taxes. \cite{tilly1985} argues that the main reason for states to collect taxes is to be able to wage war. In other words, fiscal capacity and state power are tightly linked --- so if conflict selects for state power, then it also selects for fiscal capacity. \cite{gennaioli2015} and \cite{cantoni2023} show that conflict drove state consolidation. \cite{dincecco2012} show that fiscal capacity had a positive effect on growth.

\cite{diamond1997} and \cite{landes1969, landes2006} have argued for the economic benefits of political fragmentation and state competition. \cite{diamond1997} asserts that geography 
remoteness of China deprived it of the benefits of state competition that European states enjoyed throughout their history. \cite{landes2006} makes a similar argument, pointing out how the totalitarianism and bureaucratism that stifled the spread of innovation in medieval China would not have survived long in a more fragmented state system. Geography and fragmentation also play central roles in \cite{cervellati2022}'s theory of European state history.

The closest relative of this paper is \cite{schoenholzer2022} who investigate how switching between states affects the development of cities. They find that switching states affects population negatively in the short run but positively in the long run. Further, they show that this long-run benefit is driven by improved state quality after switching. Similar to the evolution metaphor, they interpret this as "creative destruction": the short run cost of destruction is compensated by the long-run benefit of becoming part of a higher-quality state. However, they do not show explicitly how these improvements in state quality are driven by selection through conflict. Their data has a bigger geographical scope (all of Europe) but lower resolution: they use urban population from \cite{bairoch1988} as a proxy for development, which is available for fewer cities and fewer years.

The analysis in this paper builds upon that of \cite{schoenholzer2022} and extends it by identifying special cases of switching events: in \textit{conquests} a city is forcefully absorbed into a new state, while in \textit{successions} a city is absorbed into a new state after the rulers of the old state have gone extinct. Comparing the effects of these two types of switches allows me to tease out the effect of state power. Identification is based on the economic assumption that conquests imply state power differentials but successions do not. Specifically, if we consider the universe of all switching events, the average increase in state power from the old to the new owner would be higher for the conquests than for the successions. The reasoning is that more state power is necessary to conquer a city, which generally involves winning an armed conflict, than to inherit it from an extinct lineage, which mainly requires favourable marriages. This assumption allows me to use conquests as a stand-in for an increase in state power. Additional assumptions are needed to interpret my results causally.

While I am able to replicate \cite{schoenholzer2022}'s result that switching to another state has a positive effect one century later, I fail to identify a statistically significant effect of state power on urban development. This does not necessarily mean that conflict-driven evolution played no role in the rise of Europe --- there are fundamental issues with the data and the approach which limit the statistical power of my analysis. The most important one is attenuation bias from measurement error in the independent variable, and I argue that the effect may well be significant in its absence.

The rest of the paper proceeds as follows. Section \ref{sec:background} gives essential historical background information. Section \ref{sec:data} describes the data used in the analysis. Section \ref{sec:empirics} details the empirical approach. Section \ref{sec:results} presents results. Section \ref{sec:discussion} discusses implications and robustness of the results. Section \ref{sec:conclusion} concludes.


% ######################################################

\section{State system of the HRE} \label{sec:background}

The Holy Roman Empire was a powerful monarchy that existed in Central Europe from the Middle Ages until its dissolution in 1806. Though ruled by the Holy Roman Emperor at the Imperial level, it consisted of many small and large territories with their own regional rulers. Before Germany first became a republic in 1918, it is not helpful to think of "states" in the modern sense: while territories may have existed as political entities on paper, they were ruled by aristocrats who handled power in their familial structures \citep[pp. 2-3]{pt2}. These aristocratic families defined the territorial history of the HRE: territories were split by inheritances, joined by marriages and extinctions, and driven to war by struggles for dominance among the noble lineages ruling them. For this reason, when I say "state" in the context of this paper I mean its early modern German equivalent --- the territory of a lineage of rulers, or the lineage itself.


\section{Data} \label{sec:data}

I use data from the \textit{Princes and Townspeople} collection which is based on the \textit{Deutsches Städtebuch} \citep{keyser1939} and \cite{kobler2007}'s \textit{Historisches Lexikon der Deutschen Länder}. Out of the six datasets currently published in the collection, I use those on territorial histories \citep{pt2}, construction activity \citep{pt5} and conflict incidents \citep{pt6} in my analysis.\footnote{I thank Davide Cantoni and Matthias Weigand for giving me access to unpublished yearly versions of the construction and territorial histories data.} I also use the geo-referenced city locations \citep{pt1} to create visualisations. 

From these sources I assemble a panel of cities. City-periods are the unique units of observation. The ultimate dataset contains, in abstract terms, two types of variables: ones that code the territorial history of cities, and ones that code the incidence of construction and conflict events. Records for each city begin in 1300 at the earliest and end at the latest in 1800 (after which no construction events are recorded). The median city is observed over a time span of 290 years.

Figure \ref{fig:map_cities_raw} shows the spatial distribution of the 2,390 cities in the original \textit{Princes and Townspeople} data. The reporting boundaries are those of the German Empire in 1937 plus the Free City of Danzig; within those boundaries \cite{keyser1939} claim to list "every settlement that at one point in the history of the Holy Roman Empire was awarded the status of a city" \citep[p. 2]{pt1}.

\paragraph{Territorial histories.} 

A city-by-year panel of territorial history, taken from \cite{pt2}, forms the backbone of my dataset. For each city and each year it identifies the noble lineage to whose territory the city belonged in that year. I use this information to find years in which cities switch from one territory to another. The median city in the original dataset switches territories 5.64 times in its history. 

The original data also records how each city came into the possession of the lineage ruling it. I combine the original 13 types of switches into 3 types: \textit{Conquests}, \textit{successions}, and \textit{other.} I code a switch as a conquest if the original data records it as either "conquest" or "acquisition by conflict", and as a succession if the record says "extinction of lineage" or "inheritance". The remaining categories are classified as \textit{other}.\footnote
{
The other categories are: "no data or regular status", "personal union or marriage", "purchase", "foundation", "secularisation", "decision of the emperor", "gift", "exchange" and "mediatisation".
} 
On average, cities spend 16\% of their recorded lifespan under the rule of states that conquered them, 34\% under states that inherited them from an extinct lineage, and 50\% under states that gained possession of them in another way.

\paragraph{Construction activity.}

The data on construction activity \citep{pt5} and conflict incidents \citep{pt6} both resemble lists of events. The unit of observation is the event, occurring in a specified city at a specified time point. Each entry contains additional information about the type of event and the range of uncertainty for the timing. Because I drop city-period observations with missing construction data from the sample, it is worth outlining how I transform a list of events into a city-period panel of construction activity. First I group and count events that occur in the same city and period. Then I fill the observations in between the events with the value zero. Finally, I code values as missing for periods that are earlier than the first recorded event or later than the last recorded event in a city. I do not set these out-of-range values to zero because I have no way of knowing whether nothing actually happened, or whether the record simply does not cover these periods anymore.

I will regress construction activity on breaks in the territorial history of cities throughout the main part of this paper --- this is not self-explanatory. After all, the research question deals with the effect of switching states on urban economic development. I use construction as a proxy for development because it is available for most cities: out of the 2,390 cities in the \textit{Städtebuch}, 2,177 have at least one construction event linked to them. By contrast, only 207 cities from the \textit{Städtebuch} appear in \cite{bairoch1988}'s dataset of city populations.

The historical record of construction events is unfortunately not exhaustive and mostly contains notable buildings --- churches, palaces, castles, malls, schools, libraries, opera houses, gardens, and the like. It contains residential buildings of some noble families and notable citizens but not of the general population. Even so, there are theoretical justifications for using construction activity as a proxy for growth: If buildings are a normal good, then demand for them should increase with income. While the types of buildings listed do seem to represent state and elite consumption more than general economic activity, the income of these groups is also dependent on the broader economy --- an upturn will increase tax revenues and by extension the disposable income of the state. On a basic level we would expect more new buildings, of any type, in cities that are booming (although the error term is likely large - in the short run we might also expect more new buildings in cities that were destroyed by war or environmental catastrophes).

In an attempt to understand more about the effects I estimate, I use certain subsets of construction activity as the outcome in alternative specifications. There are 16 types of construction activity in the original data, which I combine into 3 facets: \textit{State construction} consists of palaces, castles, administrative and military buildings. \textit{Private construction} includes malls, economic (banks, factories, restaurants, and more) and private (residential) buildings. Lastly, \textit{construction of public goods} includes infrastructure, social, cultural, and educational buildings, memorials, gardens and parcs.\footnote
{
Churches, monasteries, clerical buildings and town halls are not included in any of the custom subsets of construction activity. They are, however, included in the "All construction" specifications.
}


\paragraph{Conflict incidents.}

I include conflict incidents (sieges, plunders, civil unrest) as a control variable in a robustness check. Conflict in my dataset is simply coded as an indicator whether a conflict incident of any type is recorded in this city and period. I code missing values exactly as I do for construction events.

Note that conflict \textit{incidents} are distinct from conflict as a driver of improvements in institutions. Conflict incidents are events recorded at the city level that often involve the destruction of life or property in that city. By contrast, in the hypothesis "conflict selects for states with higher-quality institutions" I mean conflict at the macro level, as a process that determines the fates of entire territories.

\paragraph{Aggregation.}

I aggregate this yearly data into periods with a length of 50 years. This is primarily because many events (conflict as well as construction) are recorded with uncertain timing, i.e. descriptors like "beginning of the 16th century" rather than an exact year --- I can use such events in the aggregated version if their timing can be assigned with certainty to a 50-year period. A pleasant side effect of using aggregated data is that coefficient plots and tables are easier to read.

The switch indicators are aggregated by summing, to obtain the total number of switches that happen in a period, and the aggregated indicators for construction and conflict take the value 1 if at least one event was recorded during the period. 

\paragraph{Accounting for dropped observations.}

The raw \textit{Territorial Histories} dataset starts out with 1,471,934 yearly observations, of which I lose a small number while coding switches. I aggregate the rest into 30,980 observations with 50-year spacing. Of these I drop 16,110 that are not covered by the construction data. Finally, following \cite{schoenholzer2022}, for the baseline versions of all specifications I restrict the sample to cities with no more than two lifetime switches.\footnote
{
This is to avoid creating an excessive amount of duplicate observations. Allowing more switches skews the sample towards cities that switch often. Keeping all cities and using only e.g. the first switch for each city skews it in favour of early switches. I relax this restriction in a robustness check.
} 
This removes another 9,975 observations, leaving 6,635 in the baseline sample.

Figure \ref{fig:map_cities_sample} plots the spatial distribution of the cities that make it into the baseline sample. The total number of cities has shrunken visibly compared to Figure \ref{fig:map_cities_raw} --- we can also see, however, that the shrinking appears to have been roughly uniform, preserving the proportions between regions. Only Brandenburg (containing Berlin) and Silesia (containing Breslau) seem to lose disproportionately many observations.

% ######################################################

\section{Regression equations} \label{sec:empirics}

\subsection{Dynamic effects of switching to a new state}

First I attempt to replicate one of the key results of \cite{schoenholzer2022}: that switching states had a positive effect on the growth of cities in the long run. This is an important piece of evidence in support of my hypothesis: states that have successfully taken territory from another are "higher quality", i.e. better for growth. \cite{schoenholzer2022} arrive at this result using an event study with staggered adoption.\footnote
{
    Equation (3), \citealp{schoenholzer2022}, p. 12.
}
The outcome is city population, recorded in intervals of 50 to 100 years. The treatment is switching to another state from one observation to the next. Importantly, a city may change hands multiple times between observations but belong to the same state in the end --- such an event does \textit{not} count as treatment. As the authors show throughout their paper, the short-run effect of switching (either permanently or intermittently) is distinct from the long-run effect of permanently switching to a different state.

I will borrow extensively from this framework. The first regression I run is almost identical to the one described above:

\begin{equation}
\label{eq:sw22}
    construction_{it} = \alpha_i + \delta_t + \mathbf{1}(S_{it} > 0)\beta + 
    \sum_{\tau = -100}^{200} \mathbf{1}(t = e^{NewState}_i + \tau)\pi_\tau + \varepsilon_{it},
\end{equation}

where the outcome $construction_{it}$ is an indicator that equals 1 if any construction activity was observed in city $i$ and period $t$, the variable $S_{it}$ is the number of times a city switched states in period $t$, the variable $e^{NewState}_i$ denotes the first period in which city $i$ belongs to a different state than the previous, and $\alpha_i$ and $\delta_t$ are city and period fixed effects. The coefficients of interest are the dynamic effects $\pi_\tau$ of switching to a new state permanently. 

This is a standard TWFE regression with staggered adoption and dynamic effects. Treatment events are \textit{lasting} switches to another state, i.e. a switching event only counts as treatment if the city belongs to a different state at the beginning of the next period.\footnote
{
    If a city undergoes multiple lasting switches, I create duplicate entries for each different $e^{NewState}_i$.
}
This is because I want the coefficients $\pi_\tau$ to capture the effects of the new state's governance, not just of the political and economic turmoil that may be associated with a transition of power. I separately control for other switches by including the term $\mathbf{1}(S_{it} > 0)$ on the right-hand side so that the coefficient $\beta$ absorbs the effect of switching states in the same period.

Since the outcome is binary in this case, we can interpret Equation \eqref{eq:sw22} as a linear probability model: we model $construction_{it}$ as a random variable that follows a Bernoulli distribution with a probability parameter $p$ that is conditional on the right-hand side variables. This way we can interpret the values of coefficients as differences in the probability of construction in a given period. The coefficient $\pi_\tau$ for given $\tau$ is the difference in the probability of construction between two cities $i$ and $j$ with identical time-invariant characteristics, in the same period, who undergo the same kind of switching turmoil in this period. The only difference is that one city switched permanently to another state $\tau$ periods ago, while the other did not.

% Since comparisons between cities at different \tau are problematic, we also use the Sun and Abraham estimator (as do SW22).

Based on the theory of conflict-driven evolution we would expect any such difference to stem from the fact that switching states generally means switching to a higher-quality state (due to selection). As such, even though it would be possible in principle to include state fixed effects, it is not a good idea here: state fixed effects would absorb the effect of differences in state quality which we want to capture in $\pi_\tau$ \citep[p. 13]{schoenholzer2022}.

The basic assumption that needs to be satisfied for a causal interpretation of TWFE estimates is parallel trends. This means that, absent treatment, the outcome in cities that received treatment would have followed the same time trend as the outcome in never-treated cities. While it is impossible to empirically prove this assumption (because we cannot observe the counterfactual trend), it is common to check its validity by estimating pre-treatment effects: if the treatment and control group already showed different trends before being treated, there may be an unobserved variable correlated both with receiving treatment and with the outcome, biasing the estimate. For this reason the regression also includes a lead reaching 100 years into the past.


\subsection{Heterogeneous effects of switching}

In the best case, if the identifying assumption is satisfied, the estimates from the previous regression tell us what the effect of switching states is --- however, they cannot tell us why. In the following I attempt to estimate the effect of state power, or more accurately, the effect of the difference in selection for state power between conquests and successions, on construction activity. 

Because the \textit{Städtebuch} data compiled by \cite{pt2} records exactly how each city was acquired by the territory holding it, I am able to estimate heterogeneous effects for different types of ownership changes.\footnote
{
Presumably \cite{schoenholzer2022}, having sourced their data on territorial histories from historical maps, lack detailed information about individual switching events. They instead use clever panel decomposition methods (pp. 14-20) and data on state-level characteristics (pp. 21-23) to explain what drives their estimates.
}
For this purpose I extend Equation \eqref{eq:sw22} by interacting the treatment indicator with the type of switching event:
\begin{equation}
\label{eq:baseline}
\begin{split}
    construction_{it} &= \alpha_i + \delta_t + \mathbf{X}_{it} \beta + 
    \mathbf{1}(t \ge e^{NewState}_i) \pi \\
    &+ \mathbf{1}(t \ge e^{NewState}_i) \times
    \mathbf{1}(Conquest_{it}) \cdot \theta_{conquest} \\
    &+ \mathbf{1}(t \ge e^{NewState}_i) \times
    \mathbf{1}(Other_{it}) \cdot \theta_{other} + \varepsilon_{it},
\end{split}
\end{equation}

For clarity I have moved from the event study to a basic difference-in-differences specification; this way there is only a single coefficient of interest. I simply replace the lead and lag dummies with the post-treatment indicator $\mathbf{1}(t \ge e^{NewState}_i)$. Another minor change is that this equation features a vector of city-year level control variables $\mathbf{X}_{it}$ --- however, in the baseline version of this specification I control only for the dummy $\mathbf{1}(S_{it} > 0)$ that indicates switches in the same period, exactly like in Equation \eqref{eq:sw22}. In a robustness check I will additionally control for conflict incidents.

The more important novelty are the interaction terms: I interact the post-treatment indicator with the type of switch to obtain estimates of how the trends in construction activity differ after different types of switches. I distinguish three categories: $Conquest$, $Succession$, and $Other$. In all cases this refers to how the state currently owning the city came into power --- for example $Conquest_{it} = 1$ means that the state that holds city $i$ at time $t$ acquired city $i$ by conquest.

A key economic assumption is needed, even before the econometric ones, for these interaction terms to tell us anything about state power: the power of the state owning city $i$ is correlated with the manner in which the state acquired city $i$. In particular, I assume that conquests are more strongly positively correlated with state power than successions. Support for this assumption comes from \cite{levine2021}'s model of conflict-driven evolution. BLANK. I will discuss its validity at length later on.

If this assumption holds true, the effect of state power should manifest itself in the difference between the trend of cities acquired by conquest and the trend of cities acquired by succession. Note that successions are the omitted category --- there is no interaction term between the post-treatment indicator and $\mathbf{1}(Succession_{it})$ in Equation \eqref{eq:baseline} --- so that the coefficient $\theta_{conquest}$ captures exactly what I want to estimate: the difference in trends after conquests vis-a-vis successions.

The econometric assumption that must hold true for $\theta_{conquest}$ to be a causal effect is a different version of the parallel trends assumption. In particular, it must hold that, had a conquered city $i$ counterfactually been acquired by succession rather than conquest, it would have trended the same as the other cities acquired by succession. BLANK.


% ######################################################

\section{Results} \label{sec:results}

\subsection{Construction activity increases after switching states}

Table \ref{tab:SW22_replication_50y} contains estimates for the coefficients $\pi_\tau$ from Equation \eqref{eq:sw22}. Each column reports the effect of switching states on a different subset of observed construction activity: Column (1) for all types of buildings, Column (2) for administrative, military and representative buildings, Column (3) for private construction, and Column (4) for buildings that provide public goods. The exact definitions of these categories are as described in Section \ref{sec:data}. Note also that a $Period$ is an interval of 50 years that begins with the given year. $Period = -50$ is omitted as the reference category for treated cities. Reassuringly, as the coefficients on $Treat \times Period = -100$ show, there are no significant pre-trends in any of the outcome variables. 

We can see in Column (1) that switching states has no immediate effect on the probability of recorded construction activity. Over the next two centuries, however, the effect grows positive and significant: 100 years after a city switches states, the probability of construction activity is 5 percentage points higher than in cities that did not switch. After 200 years this difference grows to 9 percentage points. These estimates are statistically significant at the 5\% and 1\% level, respectively. Figure \ref{fig:SW_replication} plots the coefficients in Column (1) to visualise these dynamics.

The regression in Column (1) uses all construction as the outcome, while Columns (2), (3), and (4) use the subsets of construction activity defined in Section \ref{sec:data}. Almost all the coefficients in the last three columns are insignificant, suggesting that neither of these specific subsets of construction is driving the overall results. Curiously, though, 50 years after switching there is a small, transient, but significant increase in construction of public goods. Since many of these would have been provided by the state, an explanation could be that rulers are initially reluctant to invest into their new territory --- the coefficient for the 50 years immediately following the switch ($\tau = 0$) is insignificant but negative --- but catch up once their rule is established.


\subsection{No evidence for the state power hypothesis}

Table \ref{tab:baseline_50y} shows estimates from Equation \eqref{eq:baseline}. This is a difference-in-differences specification with added interactions to capture heterogeneous treatment effects. As before, different columns use different subsets of construction as the outcome variable. Since successions are omitted from the interaction terms, the coefficient $\pi$ on the post-treatment dummy $Switch\,to\,another\,state$ captures the treatment effect of switching states via succession. The interaction terms are relative to that: the coefficient $\theta_{conquest}$ on $Switch\,to\,another\,state \times Conquest$ is the difference between the treatment effect of switching via succession, and the treatment effect of switching via conquest.

In Column (1) we can see that the probability of construction activity in a given period is 4 percentage points higher following a conquest vis-a-vis following a succession. The treatment effect of a conquest (compared to no treatment) is $0.02 + 0.04 = 0.06$, while the treatment effect of a succession is $0.02$. Crucially, though, the difference between them is not statistically significant. Given that my assumption about the difference in selectivity of conquests and successions holds true, we would have expected the treatment effect of conquests to be significantly higher.

The only subset of construction for which the difference in trends is indeed significant is state construction --- palaces, castles, military and administrative buildings. We can see in Column (2) that these are 5 percentage points \textit{less} likely to be built in a city after it was conquered; the effect is significant at the 5\% level. BLANK.


% ######################################################

\section{Discussion} \label{sec:discussion}

The results in Table \ref{tab:SW22_replication_50y} corroborate part of my hypothesis: switching states within the Holy Roman Empire had a positive effect on the development of cities in the long run. However, the null result in the rest of the analysis fails to convince us that conquests drove the positive overall effect. I find weak evidence at best for the hypothesis that selection for state power played a role in the economic development of European cities. In this section I check whether the null result in Table \ref{tab:baseline_50y} is robust to alternative specifications and discuss its meaning.

As a first check I include different controls $\textbf{X}_{it}$ in Equation \eqref{eq:baseline}. Table \ref{tab:controls_50y} shows the results of this exercise. Column (1) is identical to Column (1) in Table \ref{tab:baseline_50y} and controls only for switching in the same period. In Column (2) I remove this control. Columns (3) and (4) add conflict incidents as a control variable, once in addition to switching and once without. Finally, Column (5) runs the baseline specification from (1) on the smaller sample used in (3) and (4). The outcome is $All \, construction$ in all regressions. 

Comparing Column (2) to (1), and (4) to (3), we can see that it makes no difference whether we control for switches in the same period or not; the estimates are barely distinguishable. When we control for conflict incidents, however, the coefficient $\theta_{conquest}$ suddenly turns negative, highly significant and large in magnitude. After a conquest the probability of observing construction in a given period now increases by 15 percentage points \textit{less} than after a succession.

While this result may seem like evidence for omitted variable bias in the baseline estimate, that explanation is ultimately not plausible. Column (5) shows that the disagreement between the estimates does not come from controlling for conflict, but from the sample composition: more than two thirds of observations are lost between Columns (2) and (3) because of missing values in the $Conflict$ variable. When I run the baseline specification on this partial sample, I obtain the exact same estimates as in Column (3) --- even without controlling for conflict.

Figure \ref{fig:conflict_map} shows that the observations that remain in the partial sample follow quite a different distribution across space --- if the spatial composition of the partial sample was the same as that of the original sample, the entire map would be a uniform shade. Instead we see big discrepancies: the west, the south, and Brandenburg retain 30\% to 60\% of their observations, while many northern and eastern regions all but vanish from the sample.

Together, Table \ref{tab:controls_50y} and Figure \ref{fig:conflict_map} have the interesting implication of spatially heterogeneous differences in treatment effects. The null difference in Column (1) could be an average of the negative difference in the southwest and a hypothetical positive difference in the northeast. Outsiders could play a role, in a similar vein to \cite{levine2021, levine2022}'s theory: the south and the west are bordered by the Habsburg Empire and France, two stable major powers, while the northeast is somewhat sheltered by the North Sea, the Baltic Sea, and the territory that comprises 1937 Poland, which was fragmented throughout much of its history.\footnote
{
Of course, England and Sweden would attack via the North and Baltic seas, but it seems a reasonable assumption that competition for territory would be more intense between states that share a land border.
} 
BLANK \cite{cervellati2022}. I leave this for future research to explore.

I conduct a number of other robustness checks, which are summarised in the Appendix. Table \ref{tab:robustness_50y} contains regressions using different definitions of the sample and treatment. Columns (2) and (3) use larger samples that include cities with a higher number of lifetime switches. Column (4) shows the results of a regression in which the post-treatment dummy reverts back to 0 after 200 years. With this I intend to replicate the same event window as in Equation \eqref{eq:sw22} and to zero in on the effect over the first two centuries following a switch. While increasing the cutoff for lifetime switches increases the significance of the overall treatment effect (likely because the sample contains many duplicates of cities that switch often), none of these alternative specifications give the coefficient we care about statistical significance.

\decoline

To summarize these results: I find robust non-evidence for the hypothesis that selection for state power drove city growth in the Holy Roman Empire between 1300 and 1800. The question is, can we trust this non-finding?

The key economic assumption underlying the identification strategy for the effect of state power was that states that acquire a city by conquest are more powerful on average than states that acquire a city by succession. If this assumption does not hold, then the difference between the treatment effect of a conquest and the treatment effect of a succession is just that: a heterogeneity with no obvious meaning.

I justify this assumption with the fact that a conqueror is, by definition, the winner of an armed conflict. State power is not the only determinant of winning armed conflicts, though: luck plays a central role, as do many other time-varying factors for which I cannot control. On the flip side, winning armed conflicts is also not the only consequence of state power. Importantly, the rulers of more powerful states may be able to make more advantageous matches for their children --- dynastic marriages that eventually result in the expansion of their territory. Exhibit A in this argument is the House of Habsburg: they became a dominant power in Europe at the turn of the 16th century when strategic marriages made them rulers of Burgundy and Castille, but Habsburgs had been the Holy Roman Emperors for decades prior.

These arguments could imply that the difference between conquering and inheriting a city does not really capture state power differentials --- that my measure does not actually contain much information about the independent variable. If conquests are at best a noisy measure of state power, we essentially have measurement error in the independent variable. Accordingly, we should expect attenuation bias to push the result towards zero. The estimate for $\theta_{conquest}$ has the expected positive sign in my baseline regression --- it just lacks statistical significance. If we could regress differences in construction trends on directly observed differences in state power, we would sidestep this measurement error and likely find larger and significant effects. 

Econometrics come on top of this basic assumption about the interpretation of the independent variable. We can interpret the interaction coefficient $\theta_{conquest}$ as a causal effect if the following holds: had a conquered city $i$ counterfactually been acquired by succession rather than conquest, it would have trended the same as the other cities acquired by succession. 

This assumption requires that there are no uncontrolled-for variables that are correlated with being conquered and also have an effect on construction activity (while not being outcomes of state power themselves). Note that I control for all time-invariant characteristics of cities in the city fixed effect. However, there is one thing that comes to mind: maybe switching to another state had a more positive effect on a city's development if no other cities switched to the same state at the same time. A single city may be more easy for the new state to integrate into its economy than many, or alternatively, rulers who win multiple cities at the same time may invest only in their favourite cities and neglect the rest. 

An entire territory goes to another state (or multiple) when a ruling lineage goes extinct, but conquests usually happen one at a time. The average number of simultaneous switches is likely much larger for successions than for conquests. If conquests had an effect on construction activity not just through the change in state power but also through the number of cities that switched simultaneously, then the coefficient $\theta_{conquest}$ no longer captures only the causal effect of switching. I believe this is one of the remaining threats to the internal validity of my regression analysis. It should be possible to compute the number of cities that switch to the same state as city $i$ in period $t$ and then to control for it in Equation \ref{eq:baseline}, which would solve the problem. I leave this for a future extension of this paper.

Having discussed issues with internal validity now, I will briefly talk about external validity. We cannot infer from my results that the theory of conflict-driven evolution is true or false. I test a general prediction of the theory in the Holy Roman Empire between 1300 and 1800, which as a setting is fairly unique. It features a vast array of fragmented states that compete and go to war with each other, but are also part of the same empire and likely to share many institutions. Religious conflict peaks in the sample period, with the Thirty Years War taking place from 1618-1648, and alliances are often based on the shared confession of rulers. Spain, by contrast, was a large monolithic state by the end of the Reconquista in 1492, in which the Protestant reformation never really took off.

Based on these unique features of the setting we should be wary even of generalising my results to the rest of Europe.\footnote
{
\cite{schoenholzer2022} fill this gap partially, given that their sample comprises all of Europe and they at least find a similar overall effect of switching states on long-run development.
}
Nonetheless, the results have value outside of understanding the history of the Holy Roman Empire: as a puzzle piece. A main objective of this paper was to test a theory. Theories become more widely accepted when researchers repeatedly test and confirm their predictions; they are discredited when their predictions stubbornly disagree with reality. Even the result that there is no evidence for the prediction is valuable because it points towards limitations of the theory.




% ######################################################

\section{Conclusion} \label{sec:conclusion}

Early modern Europe was a uniquely war-torn place. A recent theory models wars and conflict as a Darwinian selection mechanism: strong states absorb the territory of weak ones, which gradually filters out weaker states. If this is true, territory that changes hands during conflict will generally tend to move into the possession of a stronger state and subsequently benefit from its strength. 

In this paper I test this prediction of the theory in the setting of the Holy Roman Empire. I want to understand the role that the process of conflict-driven evolution played in the economic development of Europe. I use data compiled from an encyclopedia of 2,390 historically German cities, covering the years 1300 to 1800 and featuring records of construction activity as a city-level proxy for economic activity. I run panel regressions with two-way fixed effects and staggered adoption on this dataset. 

First I estimate the dynamic effects of switching to another state. A significant effect manifests one century after a switch and becomes larger over time; after two centuries, we are 9 percentage points more likely to observe construction in cities that switched states. I do not find a pre-trend.

I then estimate heterogeneous treatment effects of different types of switches. In particular, I care about the difference between the treatment effect of conquests and the treatment effect of successions. If we assume that conquered cities tend to move into the possession of stronger states than inherited cities, we can interpret the difference in treatment effects as the effect of state power. The literature suggests that state power is positively linked to growth-fostering institutions like fiscal capacity; we would therefore expect it to have a significant positive effect on economic (construction) activity.

I indeed obtain a positive estimate for the heterogeneity --- conquests increase the probability of observing construction activity by 4 percentage points more than successions --- but it is not statistically significant. This finding remains unchanged in most robustness checks. 
However, because the contrast between conquests and successions with respect to state power is not very sharp, all these estimates likely have attenuation bias working against them. Accounting for this possibility makes the null result seem less unexpected, and more like suggestive evidence: it means that the coefficient would be bigger and possibly significant if I had a more accurate measure of state power to regress on. 

The main takeaway is that the effect of switching to another state on urban development is positive in the long run, and that I find tentative empirical support for the theory of conflict-driven selection.

It would be a big improvement on my research strategy to find a clearer measure of state power, to check whether measurement error really does bias my estimates downward. \cite{cantoni2023} construct a novel dataset of territory-level institutions which one could potentially use for that purpose. Another improvement would be to control for the number of cities switching to another state at the same time in Equation \eqref{eq:baseline}, as described in the Discussion. Third, an event study version of Equation \eqref{eq:baseline} may be able to better tease out significant effects if the effect of state power takes some time to phase in. Finally, recent literature (\cite{baker2022} provide an overview) has questioned the validity of standard two-way fixed effects regressions with staggered adoption and time-varying treatment --- one could improve my results by using one of the robust estimators proposed by \cite{sunab2021} or \cite{cengiz2019}, for example.

Future research in the setting of my thesis could explore the spatial heterogeneity for which I find evidence in Table \ref{tab:controls_50y} and Figure \ref{fig:conflict_map}. It would be interesting to learn why the coefficient on the conquest interaction is so dramatically different when the sample is skewed towards the southern and western regions. One could also take a much deeper dive into \cite{levine2021}'s theory and use the detailed geospatial and territorial histories data to test other implications of their model.

While the research question may not seem hugely relevant for the present-day Western world, I want to close, briefly, with the Democratic Republic of the Congo (DRC). This African state's central government is weak and large parts of its territory are controlled by independent armed groups \citep[p. 34]{sierra2020}. Recent research has studied the DRC to investigate questions about state formation \citep{sierra2020}, taxation and state capacity \citep{balan2022}, and conflict \citep{henn2021}. One could argue that we are witnessing economic history "in real time" there.\footnote
{
This thought is not my own; Davide Cantoni said it to me in conversation.
} 
My point is that a better understanding of how the European state system evolved may not only deepen our knowledge of the origins of institutions --- it could also inform our perspective on regions of the world that do not have a stable state system at present.

\newpage
\onehalfspacing
\bibliographystyle{chicago}
\bibliography{references.bib}

\newpage
\section*{Tables and figures}



\subsection*{Figures}

\begin{figure}[ht]
    \centering
    \includegraphics[scale=0.15]{paper/output/descriptive/map_cities_raw.png}
    \caption{Distribution of cities in the \textit{Territorial Histories} dataset \citep{pt2}. Each dot represents a city. Some major cities are identified with labels. Polygons on the map do \textbf{not} represent historical borders; they merely represent regions by which cities are clustered in the \textit{Städtebuch}.}
    \label{fig:map_cities_raw}
\end{figure}


\begin{figure}[ht]
    \centering
    \includegraphics[scale=0.15]{paper/output/descriptive/map_cities_sample.png}
    \caption{Distribution of cities in the baseline sample. Excludes cities that underwent more than two lifetime switches, and cities without any recorded construction activity. Each dot represents a city. Some major cities are identified with labels. Polygons on the map do \textbf{not} represent historical borders; they merely represent regions by which cities are clustered in the \textit{Städtebuch}.}
    \label{fig:map_cities_sample}
\end{figure}



\begin{figure}[ht]
    \centering
    \includegraphics[scale = 0.4]{paper/output/regressions/SW22_replication_50y.png}
    \caption{Dynamic effects of switching to another state. Coefficient estimates from Table \ref{eq:sw22}, Column (1). The outcome variable is the probability that construction activity of any type is observed in the period. The period beginning 50 years before the switch is omitted.}
    \label{fig:SW_replication}
\end{figure}

\begin{figure}[ht]
    \centering
    \includegraphics[scale = 0.15]{paper/output/descriptive/map_conflict_NA_50y.png}
    \caption{Share of observations from each \textit{Städtebuch} region with non-missing data on conflict incidents. I code the $Conflict$ variable as missing for a period if it is earlier than the first recorded incident or later than the last recorded incident in a city. Observations with missing conflict data are excluded from the regressions in Columns (3) and (4) of Table \ref{tab:controls_50y}.}
    \label{fig:conflict_map}
\end{figure}

\clearpage
\subsection*{Tables}

\begin{table}[htbp]
   \caption{\label{tab:SW22_replication_50y} Dynamic effects of switching}
   \centering
   \begin{tabular}{lcccc}
      \tabularnewline \midrule \midrule
      Dependent Variables:            & All construction & State  & Private & Public goods\\  
      Model:                          & (1)              & (2)    & (3)     & (4)\\  
      \midrule
      \emph{Variables}\\
      Treat $\times$ Period $=$ -100  & 0.00             & 0.02   & -0.01   & 0.00\\   
                                      & (0.03)           & (0.02) & (0.01)  & (0.02)\\   
      Treat $\times$ Period $=$ 0     & 0.00             & 0.02   & 0.00    & -0.02\\   
                                      & (0.02)           & (0.02) & (0.01)  & (0.01)\\   
      Treat $\times$ Period $=$ 50    & 0.02             & -0.01  & 0.01    & -0.03$^{**}$\\   
                                      & (0.02)           & (0.02) & (0.01)  & (0.01)\\   
      Treat $\times$ Period $=$ 100   & 0.05$^{**}$      & 0.01   & 0.00    & 0.00\\   
                                      & (0.03)           & (0.02) & (0.01)  & (0.02)\\   
      Treat $\times$ Period $=$ 150   & 0.05$^{*}$       & -0.01  & 0.02    & 0.00\\   
                                      & (0.03)           & (0.02) & (0.02)  & (0.02)\\   
      Treat $\times$ Period $=$ 200   & 0.09$^{***}$     & 0.03   & -0.01   & -0.01\\   
                                      & (0.03)           & (0.03) & (0.02)  & (0.02)\\   
      Switching indicator             & -0.01            & 0.01   & -0.01   & 0.00\\   
                                      & (0.02)           & (0.02) & (0.01)  & (0.01)\\   
      \midrule
      \emph{Fixed-effects}\\
      City                            & Yes              & Yes    & Yes     & Yes\\  
      Period                          & Yes              & Yes    & Yes     & Yes\\  
      \midrule
      \emph{Fit statistics}\\
      Observations                    & 6,635            & 6,635  & 6,635   & 6,635\\  
      R$^2$                           & 0.3563           & 0.2799 & 0.3260  & 0.2916\\  
      Within R$^2$                    & 0.0024           & 0.0013 & 0.0007  & 0.0010\\  
      \midrule \midrule
      
      
   \end{tabular}
   
   \par \raggedright 
   Note: Table presents results of estimation equation \eqref{eq:sw22}. Yearly data was aggregated into periods of 50 years. Observations are at the city-period  level. Dependent variables are indicators that take the value 1 if  construction activity of the respective type was recorded. Standard errors are  clustered at the city level. *, **, and *** denote significance on the 10 percent, 5 percent, and 1 percent  level, respectively.
\end{table}


\begin{table}[htbp]
   \caption{\label{tab:baseline_50y} Differences in trends after switching, by switch type}
   \centering
   \begin{tabular}{lcccc}
      \tabularnewline \midrule \midrule
      Dependent Variables:                       & All construction & State        & Private & Public goods\\  
      Model:                                     & (1)              & (2)          & (3)     & (4)\\  
      \midrule
      \emph{Variables}\\
      Switch to another state                    & 0.02             & 0.01         & -0.004  & -0.04$^{***}$\\   
                                                 & (0.03)           & (0.02)       & (0.01)  & (0.01)\\   
      Switch to another state $\times$ Conquest  & 0.04             & -0.05$^{**}$ & -0.02   & 0.007\\   
                                                 & (0.04)           & (0.02)       & (0.02)  & (0.02)\\   
      Switch to another state $\times$ Other     & 0.01             & -0.03        & 0.006   & 0.03$^{*}$\\   
                                                 & (0.03)           & (0.02)       & (0.01)  & (0.01)\\   
      Switching indicator                        & -0.008           & 0.004        & -0.009  & -0.003\\   
                                                 & (0.03)           & (0.02)       & (0.01)  & (0.01)\\   
      \midrule
      \emph{Fixed-effects}\\
      City                                       & Yes              & Yes          & Yes     & Yes\\  
      Period                                     & Yes              & Yes          & Yes     & Yes\\  
      \midrule
      \emph{Fit statistics}\\
      Observations                               & 6,635            & 6,635        & 6,635   & 6,635\\  
      R$^2$                                      & 0.3553           & 0.2795       & 0.3257  & 0.2916\\  
      Within R$^2$                               & 0.0009           & 0.0007       & 0.0003  & 0.0011\\  
      \midrule \midrule
      
      
   \end{tabular}
   
   \par \raggedright 
   Note: Table presents results of estimation equation \eqref{eq:baseline}. The switch type "Succession" is omitted as the  reference category. Yearly data was aggregated into periods of 50 years. Observations are at the city-period  level. The dependent variables are indicators that take the value 1 if  construction activity of the respective type was recorded. Standard errors are  clustered at the city level. *, **, and *** denote significance on the 10 percent, 5 percent, and 1 percent  level, respectively.
\end{table}


\begin{table}[htbp]
   \caption{\label{tab:controls_50y} Heterogeneous effects - different controls}
   \centering
   \begin{tabular}{lcccc}
      \tabularnewline \midrule \midrule
      Dependent Variable: & \multicolumn{4}{c}{All construction}\\
      Model:                                     & (1)    & (2)    & (3)           & (4)\\  
      \midrule
      \emph{Variables}\\
      Switch to another state                    & 0.02   & 0.02   & -0.01         & 0.03\\   
                                                 & (0.03) & (0.02) & (0.05)        & (0.05)\\   
      Switch to another state $\times$ Conquest  & 0.04   & 0.04   & -0.15$^{***}$ & -0.14$^{**}$\\   
                                                 & (0.04) & (0.04) & (0.05)        & (0.06)\\   
      Switch to another state $\times$ Other     & 0.01   & 0.01   & 0.05          & 0.05\\   
                                                 & (0.03) & (0.03) & (0.05)        & (0.05)\\   
      Switching indicator                        & -0.01  &        & -0.09         &   \\   
                                                 & (0.02) &        & (0.06)        &   \\   
      Conflict                                   &        &        & 0.01          & 0.01\\   
                                                 &        &        & (0.03)        & (0.03)\\   
      \midrule
      \emph{Fixed-effects}\\
      City                                       & Yes    & Yes    & Yes           & Yes\\  
      Period                                     & Yes    & Yes    & Yes           & Yes\\  
      \midrule
      \emph{Fit statistics}\\
      Observations                               & 6,635  & 6,635  & 1,955         & 1,955\\  
      R$^2$                                      & 0.3553 & 0.3553 & 0.4245        & 0.4230\\  
      Within R$^2$                               & 0.0009 & 0.0008 & 0.0071        & 0.0045\\  
      \midrule \midrule
      
      
   \end{tabular}
   
   \par \raggedright 
   Note: Table presents results of estimation equation \eqref{eq:baseline} but using different control variables. The switch type  "Succession" is omitted as the reference category. Yearly data was aggregated into periods of 50 years. Observations are at the city-period level. Dependent variables are indicators  that take the value 1 if construction activity of the respective type was  recorded. Standard errors are clustered at the city level. *, **, and *** denote significance on the 10 percent, 5 percent, and 1 percent  level, respectively.
\end{table}


\clearpage
\section*{Appendix}

\setcounter{table}{0}
\renewcommand{\thetable}{A\arabic{table}}

\begin{table}[htbp]
   \caption{\label{tab:robustness_50y} Heterogeneous effects of switching: More robustness}
   \centering
   \begin{tabular}{lcccc}
      \tabularnewline \midrule \midrule
      Dependent Variable: & \multicolumn{4}{c}{All construction}\\
                                                 & Baseline & 4 switches   & All switches & Window \\   
      Model:                                     & (1)      & (2)          & (3)          & (4)\\  
      \midrule
      \emph{Variables}\\
      Switch to another state                    & 0.02     & 0.04$^{***}$ & 0.02$^{***}$ & 0.02\\   
                                                 & (0.03)   & (0.01)       & (0.01)       & (0.02)\\   
      Switch to another state $\times$ Conquest  & 0.04     & 0.03         & -0.01        & 0.04\\   
                                                 & (0.04)   & (0.02)       & (0.01)       & (0.04)\\   
      Switch to another state $\times$ Other     & 0.01     & -0.02        & 0.00         & 0.01\\   
                                                 & (0.03)   & (0.01)       & (0.01)       & (0.03)\\   
      Switching indicator                        & -0.01    & -0.02        & -0.01        & -0.01\\   
                                                 & (0.02)   & (0.02)       & (0.01)       & (0.02)\\   
      \midrule
      \emph{Fixed-effects}\\
      City                                       & Yes      & Yes          & Yes          & Yes\\  
      Period                                     & Yes      & Yes          & Yes          & Yes\\  
      \midrule
      \emph{Fit statistics}\\
      Observations                               & 6,635    & 12,979       & 31,754       & 6,635\\  
      R$^2$                                      & 0.3553   & 0.3664       & 0.3518       & 0.3553\\  
      Within R$^2$                               & 0.0009   & 0.0019       & 0.0003       & 0.0009\\  
      \midrule \midrule
      
      
   \end{tabular}
   
   \par \raggedright 
   Note: Table presents results of estimation equation \eqref{eq:baseline}. Column (1) uses the baseline sample, Column (2) includes cities with up to 4 lifetime switches, Column (3) includes all cities regardless of the number of lifetime switches, and Column (4) uses a post-treatment indicator that is only active in the first 200 years after the treatment. Yearly data was aggregated into periods of 50 years. Observations are at the city-period level. Standard errors are  clustered at the city level. *, **, and *** denote significance on the 10 percent, 5 percent, and 1 percent  level, respectively.
\end{table}



\end{document}
