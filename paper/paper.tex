\documentclass{article}
\usepackage[utf8]{inputenc}
\usepackage[a4paper,left=2.5cm,right=2.5cm,top=2.5cm,bottom=2.5cm]{geometry} 

\usepackage{natbib} % we need this so we can use citation and bib properly
\usepackage{enumitem} % allow you to customize your items (like margins etc.)
\usepackage{xurl} % allow URL breaks at any alphanumerical character
\usepackage{hyperref} % allows you to add hyperlink
\usepackage{amsmath} % allows you to use most mathematical features
\usepackage{float,fancyhdr}
\usepackage{amssymb}
\usepackage{setspace} % allows you to change the line spacing
\usepackage{xcolor}
\usepackage{graphicx} % we need this so we can add figures

\title{State Size and City Growth: \\Evidence from Dynastic Takeovers}
\author{Elias Hadj Ammar}
\date{January 2023}



\begin{document}
\maketitle
\footnote{I thank Davide Cantoni, Matthias Weigand, and my RA colleagues at the Chair of Economic History for helpful comments and feedback.}
Keep in mind that the title doesn't need to say everything. It needs to be catchy and it needs to be not-misleading about the core of the paper.
Alternative title ideas: 
\begin{itemize}
    \item "Good riddance? What happens to states when dynasties go extinct" - catchy, but obscure, and also not what the paper is about.
    \item "What kind of ruler makes states prosper?" "The best kind of autocrat?"
    \item ``Territory size, leaders and economic development: Evidence from German dynasties'' {Leaders, territory size and growth: Evidence from extinctions of German dynasties}

    \item "How individual dynasties in the HRE promoted or hindered growth" - to the point, but detached from literature.
    \item Good riddance
    \item Dynastic Takeovers
    \item Extinctions of Noble Families
    \item Preview of results
    \item Which country you belong to / ``When cities change hands''
    \item {Extinctions of dynasties and changes in development}
    \item {When Cities Change Hands: Extinctions of Dynasties and Changes in Development} - very clean but also detached from literature (may be fine, though)
    \item Family Affair / Family Matters / Family Business / something more fitting about families
    \item Successions and success
    \item {Family Matters: Extinctions of Dynasties and Changes in Development}
    \item something about game of thrones / a silly pun
    \item Adoptive Family / Foster Father or Evil Stepdad / In-laws
    \item Foster family or evil in-laws [only if many takeovers are by inlaws]
    \item ... of N German cities
    \item {When Cities Change Hands: Extinctions of Dynasties and \\Breaks in the Development of N German Cities}
\end{itemize}

\newpage

\setcounter{page}{1}
 
\section{Introduction}

Abstract: I estimate the effect of state size on development using a panel of construction events in cities of the Holy Roman Empire. I exploit variation from lineage extinctions, which caused cities to become part of different lineages' territorial holdings. I find that X. I discuss economic mechanisms that explain this effect.

motivation for effects of size 

background on the hre / motivation for using city-level data and territory switches

I use a generalised difference-in-differences framework with varying intensity of treatment. The treatment is the size difference between the old and the new rulers' territorial holdings. I will argue that the takeover constitutes a shock to territory size that is exogenous to development. I support this argument using an event study, showing that cities with different takeover experiences did not differ in growth (conditional on city and year fixed effects) before their respective takeovers. 

The outcome variable used throughout the analysis is yearly construction activity. I go to some lenghts to show that this is a good proxy for growth, comparing the present data from the Deutsches St\"adtebuch to the craftsmen real wage time series by Allen for available cities. The two measures generally align.

Results indicate that territory size has a large, significant and positive effect on city growth. Cities that joined a twice-as-big territory experienced 10 \% higher construction activity in subsequent years. Splitting the sample into cities from big, medium-sized and small territories reveals that the effect is mostly driven by cities from small territories. However, cities in big territories generally changed hands less often, making it more difficult to find a meaningful and significant effect in that subsample.

The literature says that benefits of larger states include free trade within a larger area. Trade within the HRE was not exactly free. I'm not currently aware of data on historical trade flows between German cities. However, I use the establishment of historical markets (Cantoni et al) and the presence of Jews (data from Voigtl\"ander and Voth) to investigate this channel. I find that trade increases (cities are (insignificantly) more likely to establish markets, and the presence of Jews increases) when cities join larger territories.

\newpage
\bibliographystyle{chicago}
\bibliography{references.bib}




\end{document}
