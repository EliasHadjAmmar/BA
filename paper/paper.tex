\documentclass{article}
\usepackage[utf8]{inputenc}
\usepackage[a4paper,left=2.5cm,right=2.5cm,top=2.5cm,bottom=2.5cm]{geometry} 

\usepackage{natbib} % we need this so we can use citation and bib properly
\usepackage{enumitem} % allow you to customize your items (like margins etc.)
\usepackage{xurl} % allow URL breaks at any alphanumerical character
\usepackage{hyperref} % allows you to add hyperlink
\usepackage{amsmath} % allows you to use most mathematical features
\usepackage{float,fancyhdr}
\usepackage{amssymb}
\usepackage{setspace} % allows you to change the line spacing
\usepackage{xcolor}
\usepackage{graphicx} % we need this so we can add figures
\usepackage{booktabs}

\title{Conflict Driven Evolution and Urban Development: \\ Evidence from the Holy Roman Empire\footnote{
I thank Davide Cantoni, Matthias Weigand, and my RA colleagues at the Chair of Economic History for helpful comments and feedback. I thank Matthias Weigand in particular for providing me with his own data and for pointing me towards the paper by \cite{schoenholzer2022}.
}
}
\author{Elias Hadj Ammar}
\date{May 2023}

\begin{document}
\onehalfspacing
\maketitle
\thispagestyle{empty}

\begin{abstract}
How does switching between states affect the long-run development of cities, and why? I study this question in the setting of the Holy Roman Empire, using historical data on construction activity and the territorial history of cities. I specifically test the hypothesis that competition between states, in which higher-quality states dominated, caused cities to benefit from permanent ownership changes. I find evidence in support of this.
\end{abstract}

\newpage

\setcounter{page}{1}
\doublespacing


% ######################################################

\section{Introduction}



The European state system was forged in conflict. No other continent at any time experienced more war than Europe in the early modern era (\citealp{voigtlnder2013}, p. 174). Wars brought death and devastation to the societies affected by them, but their effects also manifested on the political map --- they play out in a diverse ecosystem of polities that come and go and change size over time. Borders get redrawn and territory changes hands. One can indeed imagine this as an evolutionary process: states compete for resources and powerful states eventually absorb weaker ones. As a result, traits that increase state power --- the ability to win conflicts against rivals --- get selected for, while unfavourable traits are doomed to disappear \citep{levine2013, levine2021}.

In this paper I study the role of this process in the economic development of Europe. In particular, I estimate the effect of state power, the trait that conflict selects for, on urban development. If the effect is positive, then the conflict driven evolution of states may partially explain the eventual ascendancy of Europe: by selecting for military success, Europe's violent history may have accidentally produced fiscal capacity, efficient legal systems, and prospering middle classes.

The idea that war and conflict can raise the standard of living dates back to \cite{malthus1798}. He reasoned that "positive checks" on the population (war, disease, famine) lead to higher land-labour ratios and therefore higher levels of income per capita. \cite{voigtlnder2013} build upon this Malthusian mechanism and add an explanation for the incessant wars: taxes on those higher incomes filled the treasuries of the rulers, which increased their demand for war --- a luxury good ---, which in turn kept population low. 

My story is not about population but institutions (although it is compatible with Malthusianism). Explaining long-run growth with institutional factors has a long tradition in economics: North, Acemoglu, property rights, civil law, middle class. The key institutional change brought about by conflict was fiscal capacity \citep{tilly1985, dincecco2012}, or the state's ability to raise taxes. A newer strand of research focuses on fiscal capacity:  \cite{cantoni2023} 

Theory of wars and states: \cite{tilly1985} argues that war was the main reason for governments to invest in fiscal capacity. The formal theory of conflict-driven evolution is from \cite{levine2013}. 

\cite{diamond1997} and \cite{landes1969, landes2006} have argued for the economic benefits of political fragmentation and state competition. \cite{diamond1997} asserts that geography 
remoteness of China deprived it of the benefits of state competition that European states enjoyed throughout their history. \cite{landes2006} makes a similar argument, pointing out how the totalitarianism and bureaucratism that stifled the spread of innovation in medieval China would not have survived long in a more fragmented state system.

The closest relative to this paper is \cite{schoenholzer2022} who investigate how switching between states affects the development of cities. They find that switching states affects population negatively in the short run but positively in the long run. Similar to the evolution metaphor, they interpret this as "creative destruction": the short run cost of destruction is compensated by the long-run benefit of becoming part of a higher-quality state. However, they do not show explicitly how these improvements in state quality are driven by selection through conflict.

\hspace{0.5cm}

The analysis in this paper builds upon that of \cite{schoenholzer2022} and extends it by identifying special cases of switching events: in \textit{conquests} a city is forcefully absorbed into a new state, while in \textit{successions} a city is absorbed into a new state after the rulers of the old state have gone extinct. Comparing the effects of these two types of switches allows me to tease out the effect of state power. I find a positive and significant effect of state power on the long-run development of cities. This result suggests that conflict-driven evolution indeed played a role in the rise of Europe relative to other regions of the world.

I make a key assumption: conquests imply state power differentials but successions do not. Specifically, if we consider the universe of all switching events, the average increase in state power from the old to the new owner would be higher for the conquests than for the successions. The reasoning is that conflict --- as in a conquest --- selects the new state for power, but conflict-free successions, depending mainly on dynastic marriages, are more "random". This assumption allows me to use conquests as a stand-in for an increase in state power. Several other assumptions are needed to interpret my results causally, which I will discuss in detail below.

The rest of the paper proceeds as follows.

\subsection*{Other literature}


\cite{levine2013, levine2021} provide the theoretical foundation for conflict-driven selection, as well as compelling evidence for it. 
They argue that social evolution is driven by conflict. Their model finds that "evolution selects societies with high state power - the ability to prevail in conflict over rivals." (\citealp{levine2021}, p. 436).Their answer to Why Europe and not China or India: "Ghengis Khan, the cannon, and the English. Ghengis Khan accomplished two things: he brought cannons to Europe, and he depopulated Mongolia. In China the removal of the outside threat of the Mongols led - as the theory asserts - to an extractive hegemony. In Europe the advent of the cannon together with the preying of the English on the French and the Spanish resulted - as the theory asserts - in an inclusive balance of power." (p. 439). strong outsiders but primitive siege technology [cheap defensibility of cities] in India led to extractive balance of power.  


How they define state power (pp. 439-440): "the potential ability of a social organisation to resist and influence other societies. This depends on institutional characteristics of the society, including stability of the government and the ability to collect taxes and conscript soldiers, which depend in turn on the law-abidingness of citizens, the efficiency of the courts, and the overall economic strength of the society." They *assume* in their model (p.443) that state power is directly proportional to free resources (resources in excess of subsistence).


I do not study their explanation for the European equilibrium that we observe in history; rather, I take a step back by taking the state system and the process of conflict-driven evolution as given. What interests me are the consequences of this process for economic growth.
Is it, as \cite{schoenholzer2022} argue explicitly, beneficial creative destruction - does state power equal state quality? Or are powerful states detrimental after all - because they engage in more extraction? \cite{tilly1985}


Why we would expect the hypothesis to be true: state power, defined as the ability to win conflicts with rivals, consists of factors like fiscal capacity, efficient courts, law-abiding citizens (\citealp{levine2021}, pp.439-449) - institutional factors that are also widely held to be beneficial for growth (see the vast literature on institutions). Add some institutions lit here.


My central identifying assumption: conquering a city implies more state power than inheriting it.

\cite{schoenholzer2022} use a dataset with a bigger geographical scope but lower resolution: they use urban population from \cite{bairoch1988} as a proxy for development, which is only available in intervals of 50 or 100 years, and a series of political maps. 




Key motivation papers: \cite{levine2013, levine2021}, \cite{diamond1997}, \cite{schoenholzer2022}, \cite{cervellati2022}, \cite{voigtlnder2013}, \cite{tilly1985}

The idea that today's states are just the most successful protection rackets of the past already shows up in \cite{tilly1985}.

While frequent wars brought devastation over the people and destruction to property, \cite{malthus1798} and others after him have argued, frequent wars may have kept population in check, leading to higher land-labour ratios and therefore a higher level of income per capita. \cite{voigtlnder2013} argue that the relative wealth of Europe after 1350 can be explained by a feedback loop centered on this Malthusian mechanism: after the huge loss of life caused by the Black Death, which served as a catalyst, taxable incomes rose, which made rulers richer, which allowed them to afford more wars, which in turn kept population in check.

% ######################################################

\section{Background}


One paragraph about the history of Europe's state system; this is the place for Cervellati

One story about a city's territory switch

Two pithy figures, and a no-nonsense table

NO MORE THAN TWO PAGES TOTAL

% ######################################################

\section{Data}

Empirical tests have suffered from a lack of available historical data. Such a study requires at the very least three things: 1) a detailed territorial history of European cities, 2) some city-level measure of development reaching far enough into the past, and 3) some measure of state power, institutional quality, or whatever else one wishes to explain effects with. These data limitations are starting to be overcome. 

% ######################################################



\newpage
\onehalfspacing
\bibliographystyle{chicago}
\bibliography{references.bib}

\newpage
\section*{Tables and figures}


\end{document}
