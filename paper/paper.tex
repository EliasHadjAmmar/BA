\documentclass{article}
\usepackage[utf8]{inputenc}
\usepackage[a4paper,left=2.5cm,right=2.5cm,top=2.5cm,bottom=2.5cm]{geometry} 

\usepackage{natbib} % we need this so we can use citation and bib properly
\usepackage{enumitem} % allow you to customize your items (like margins etc.)
\usepackage{xurl} % allow URL breaks at any alphanumerical character
\usepackage{hyperref} % allows you to add hyperlink
\usepackage{amsmath} % allows you to use most mathematical features
\usepackage{float,fancyhdr}
\usepackage{amssymb}
\usepackage{setspace} % allows you to change the line spacing
\usepackage{xcolor}
\usepackage{graphicx} % we need this so we can add figures
\usepackage{booktabs}

\title{Dynastic Extinctions and Urban Economic Development: \\Evidence from the Holy Roman Empire\footnote{
I thank Davide Cantoni, Matthias Weigand, and my RA colleagues at the Chair of Economic History for helpful comments and feedback. I thank Matthias Weigand in particular for providing me with his own data and for pointing me towards the paper by \cite{schoenholzer2022}.
}
}
\author{Elias Hadj Ammar}
\date{May 2023}

\begin{document}
\onehalfspacing
\maketitle
\thispagestyle{empty}

\begin{abstract}
How does switching between states affect the long-run development of cities, and why? I study this question in the setting of the Holy Roman Empire, using historical data on construction activity and the territorial history of cities. I specifically test the hypothesis that competition between states, in which higher-quality states dominated, caused cities to benefit from permanent ownership changes. I find evidence in support of this.
\end{abstract}

\newpage

\setcounter{page}{1}
\doublespacing


% ######################################################

\section{Introduction}

The state system of early modern Europe was fraught with conflict. 

This paper studies a different interaction between conflict and growth. I study how conflict affected the institutional quality of states. In particular, I investigate whether wars and conflict served as a selection mechanism - whether the constant competition caused better states to grow and worse states to shrink and eventually disappear. 


These ideas are not new - 


"In conflict luck matters, but success begets success. Conquering a city or a province strengthens the winner and weakens the loser. Hence as war and invasion unfold we expect that eventually one side gets lucky enough, strong enough, and faces opponents weak enough, that it wins outright." (\citealp{levine2021}, p. 436).

The model developed by \cite{levine2013} finds that "evolution selects societies with high state power - the ability to prevail in conflict over rivals." (\citealp{levine2021}, p. 436).


Malthus vs conflict: Levine and Modica hate Malthus. They argue that social evolution is driven by conflict.

In particular: evolution selects for aggression. "we show that those societies that simply seek to defend their territory are doomed to domination by those who aggressively seek expansion. Hence successful societies such as Spain and England have survived precisely because they are aggressive." (p. 438) That is why Spain still cares about regaining Gibraltar and keeping Catalonia, and why the UK wants to keep Scotland and the Falklands.

Their answer to Why Europe and not China or India: "Ghengis Khan, the cannon, and the English. Ghengis Khan accomplished two things: he brought cannons to Europe, and he depopulated Mongolia. In China the removal of the outside threat of the Mongols led - as the theory asserts - to an extractive hegemony. In Europe the advent of the cannon together with the preying of the English on the French and the Spanish resulted - as the theory asserts - in an inclusive balance of power. Here as is often asserted the English played a key role in the industrial revolution - albeit a rather different one than is commonly assumed." (p. 439).


How they define state power (pp. 439-440): "the potential ability of a social organisation to resist and influence other societies. This depends on institutional characteristics of the society, including stability of the government and the ability to collect taxes and conscript soldiers, which depend in turn on the law-abidingness of citizens, the efficiency of the courts, and the overall economic strength of the society.

They *assume* in their model (p.443) that state power is directly proportional to free resources (resources in excess of subsistence).
"Then (unless the outsiders are very strong) the long-run stochastically stable outcome will maximise free resources. [in contrast to Malthus, where the long-run stable outcome is subsistence income.]" This is pretty much the result I also try to show. 
They do allow that not all free resources are spent on things that improve the chance of success in conflict (state power).

"The residual - profits if you will - may be turned to many uses and only some of these uses - swords not jewelry - enhance state power. Hence a profit maximising monarchy that through social norms is bound to use its profits for fighting and conflict is the type we expect to survive; a profit maximising monarchy that spends its profits on large and beautiful palaces has less of a future." (p. 444)

Levine and Modica in a nutshell: societies with higher state power prevail.


My own point, sharply: conquering a city implies state power, while inheriting it does not.

NEXT SECTION: 15.6 PROSPERITY

stylised fact to be explained in 15.6 (p.455): competition between inclusive institutions in Europe, extractive hegemony in China

competing societies in India - why did they generate little innovation?

they introduce another dimension alongside hegemony vs balance of power: societies can now be inclusive vs extractive

puzzle: "since extractive institutions generally levy higher taxes and have larger armies, if evolution is driven by conflict how come these "strong" extractive institutions do not predominate over "weaker" inclusive institutions?

model: inclusive institutions (commercial elites have upper hand) or extractive institutions (military elites have upper hand) within each society. Transfer from commercial to military elites in each period - smaller in inclusive societies.

results:
- inclusive hegemonies and heterogenous balances of power cannot be stochastically stable
- if inclusive institutions do not differ much from extractive ones, only extractive institutions are stable in general. (The commerical elites won't make much effort to defend inclusive institutions that barely make a difference). whether it's a hegemony or balance of power then depends on the strength of the outsiders.
- if the inclusive institutions are worth it for the commercial elites, but outsiders are weak, only extractive institutions are stable. if outsiders are strong, there is a balance of power, but whether it's inclusive or extractive depends on military technology: effective fortifications (defending is not costly) favour extractive institutions, while good siege technology (defending comes at a high cost) favours inclusive institutions. (pp. 459-460)

Result: 
- arrival of cannon and strong outsiders made stable inclusive balance of power in Europe
- weak outsiders favoured extractive hegemony in China
- strong outsiders but primitive siege technology [cheap defensibility of cities] in India led to extractive balance of power


- 




Institutional factors that benefit a state's economy also benefit a state's ability to win conflicts with others. Namely fiscal capacity


While frequent wars brought devastation over the people and destruction to property, \cite{malthus1798} and others after him have argued, frequent wars may have kept population in check, leading to higher land-labour ratios and therefore a higher level of income per capita. \cite{voigtlnder2013} argue that the relative wealth of Europe after 1350 can be explained by a feedback loop centered on this Malthusian mechanism: after the huge loss of life caused by the Black Death, which served as a catalyst, taxable incomes rose, which made rulers richer, which allowed them to afford more wars, which in turn kept population in check.


% ######################################################


% ######################################################


% ######################################################



\newpage
\onehalfspacing
\bibliographystyle{chicago}
\bibliography{references.bib}

\newpage
\section*{Tables and figures}


\end{document}
