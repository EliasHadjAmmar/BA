\documentclass{article}
\usepackage[utf8]{inputenc}
\usepackage[a4paper,left=2.5cm,right=2.5cm,top=2.5cm,bottom=2.5cm]{geometry} 

\usepackage{natbib} % we need this so we can use citation and bib properly
\usepackage{enumitem} % allow you to customize your items (like margins etc.)
\usepackage{xurl} % allow URL breaks at any alphanumerical character
\usepackage{hyperref} % allows you to add hyperlink
\usepackage{amsmath} % allows you to use most mathematical features
\usepackage{float,fancyhdr}
\usepackage{amssymb}
\usepackage{setspace} % allows you to change the line spacing
\usepackage{xcolor}
\usepackage{graphicx} % we need this so we can add figures
\usepackage{booktabs}

\title{Dynastic Extinctions and Urban Economic Development: \\Evidence from the Holy Roman Empire\footnote{I thank Davide Cantoni, Matthias Weigand, and my RA colleagues at the Chair of Economic History for helpful comments and feedback.}
}
\author{Elias Hadj Ammar}
\date{May 2023}

\begin{document}
\onehalfspacing
\maketitle
\thispagestyle{empty}

\begin{abstract}
I use historical data on construction activity and territorial history to estimate the effect of switching between territories on urban construction in the Holy Roman Empire. I exploit variation from dynastic extinctions, which caused cities to become part of different lineages' territorial holdings. I find that an X\% increase in the number of cities possessed by the ruling family is associated with a Y\% increase in construction activity.
\end{abstract}




\newpage

\setcounter{page}{1}
\doublespacing

% ######################################################
 
\section{Introduction}


% ######################################################

\section{Literature review}


% ######################################################

\section{Cities, lineages, and territories}


% #####################################################

\section{Empirical approach}


% #####################################################

\section{Results}


% #####################################################

\section{Validity and robustness}


% ######################################################

\section{Conclusion}


% ######################################################



\newpage
\onehalfspacing
\bibliographystyle{chicago}
\bibliography{references.bib}

\newpage
\section*{Tables and figures}


\end{document}
