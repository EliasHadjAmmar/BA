\documentclass{article}
\usepackage[utf8]{inputenc}
\usepackage[a4paper,left=2.5cm,right=2.5cm,top=2.5cm,bottom=2.5cm]{geometry} 

\usepackage{natbib} % we need this so we can use citation and bib properly
\usepackage{enumitem} % allow you to customize your items (like margins etc.)
\usepackage{xurl} % allow URL breaks at any alphanumerical character
\usepackage{hyperref} % allows you to add hyperlink
\usepackage{amsmath} % allows you to use most mathematical features
\usepackage{float,fancyhdr}
\usepackage{amssymb}
\usepackage{setspace} % allows you to change the line spacing
\usepackage{xcolor}
\usepackage{graphicx} % we need this so we can add figures
\usepackage{booktabs}

\title{Dynastic Extinctions and Urban Economic Development: \\Evidence from the Holy Roman Empire\footnote{
I thank Davide Cantoni, Matthias Weigand, and my RA colleagues at the Chair of Economic History for helpful comments and feedback. I thank Matthias Weigand in particular for providing me with his own data and for pointing me towards the paper by \cite{schoenholzer2022}.
}
}
\author{Elias Hadj Ammar}
\date{May 2023}

\begin{document}
\onehalfspacing
\maketitle
\thispagestyle{empty}

\begin{abstract}
How does switching between states affect the long-run development of cities, and why? I study this question in the setting of the Holy Roman Empire, using historical data on construction activity and the territorial history of cities. I specifically test the hypothesis that competition between states, in which higher-quality states dominated, caused cities to benefit from permanent ownership changes. I find evidence in support of this.
\end{abstract}

\newpage

\setcounter{page}{1}
\doublespacing


% ######################################################

\section{Introduction}

The state system of early modern Europe was fraught with conflict. 

This paper studies a different interaction between conflict and growth. I study how conflict affected the institutional quality of states. In particular, I investigate whether wars and conflict served as a selection mechanism - whether the constant competition caused better states to grow and worse states to shrink and eventually disappear. 


These ideas are not new - 

Institutional factors that benefit a state's economy also benefit a state's ability to win conflicts with others.


While frequent wars brought devastation over the people and destruction to property, \cite{malthus1798} and others after him have argued, frequent wars may have kept population in check, leading to higher land-labour ratios and therefore a higher level of income per capita. \cite{voigtlnder2013} argue that the relative wealth of Europe after 1350 can be explained by a feedback loop centered on this Malthusian mechanism: after the huge loss of life caused by the Black Death, which served as a catalyst, taxable incomes rose, which made rulers richer, which allowed them to afford more wars, which in turn kept population in check.


% ######################################################


% ######################################################


% ######################################################



\newpage
\onehalfspacing
\bibliographystyle{chicago}
\bibliography{references.bib}

\newpage
\section*{Tables and figures}


\end{document}
