\documentclass{article}
\usepackage[utf8]{inputenc}
\usepackage[a4paper,left=2.5cm,right=2.5cm,top=2.5cm,bottom=2.5cm]{geometry} 

\usepackage{natbib} % we need this so we can use citation and bib properly
\usepackage{enumitem} % allow you to customize your items (like margins etc.)
\usepackage{xurl} % allow URL breaks at any alphanumerical character
\usepackage{hyperref} % allows you to add hyperlink
\usepackage{amsmath} % allows you to use most mathematical features
\usepackage{float,fancyhdr}
\usepackage{amssymb}
\usepackage{setspace} % allows you to change the line spacing
\usepackage{xcolor}
\usepackage{graphicx} % we need this so we can add figures
\usepackage{booktabs}

\title{Dynastic Extinctions and Urban Economic Development: \\Evidence from the Holy Roman Empire\footnote{
I thank Davide Cantoni, Matthias Weigand, and my RA colleagues at the Chair of Economic History for helpful comments and feedback. I thank Matthias Weigand in particular for providing me with his own data and for pointing me towards the paper by \cite{schoenholzer2022}.
}
}
\author{Elias Hadj Ammar}
\date{May 2023}

\begin{document}
\onehalfspacing
\maketitle
\thispagestyle{empty}

\begin{abstract}
How does switching between states affect the long-run development of cities, and why? I study this question in the setting of the Holy Roman Empire, using historical data on construction activity and the territorial history of cities. I specifically test the hypothesis that competition between states, in which higher-quality states dominated, caused cities to benefit from permanent ownership changes. I find evidence in support of this.
\end{abstract}

\newpage

\setcounter{page}{1}
\doublespacing


% ######################################################

\section{Introduction}



The history of Europe was defined by conflict. No other continent at any time experienced more war than Europe in the early modern era (\citealp{voigtlnder2013}, p. 174). Wars brought death and devastation to the societies affected by them, but their effects also manifested on the political map --- they play out in a diverse ecosystem of polities that come and go and change size over time as the borders get redrawn and territory changes hands. One can indeed imagine this as an evolutionary process: states compete for resources and powerful states eventually absorb weaker ones. As a result, traits that increase state power --- the ability to win conflicts against rivals --- get selected for, while other traits are doomed to disappear \citep{levine2013, levine2021}.

In this paper I study the role of this process in the economic development of Europe. In particular, I estimate the effect of state power, the trait that conflict selects for, on urban development. If the effect is positive, then selection of "better" states through conflict-driven evolution may partially explain the ascendancy of Europe.

I do this by looking at the development of cities after they switch states. In a paper from which I draw heavily, \cite{schoenholzer2022} find that switching states affects population negatively in the short run but positively in the long run. I take their analysis a step further by identifying special cases of switches: in \textit{conquests} a city is forcefully absorbed into a new state, while in \textit{successions} a city is absorbed into a new state after the rulers of the old state have gone extinct. 
Using these concepts I make a key assumption: conquests imply state power differentials but successions do not. If state A takes a city from state B, 

absorbing a city by conquest is more likely if the new state is more powerful than the old one, but absorbing a city by succession underlies no such limitations - it only requires 

the new state after a conquest is selected for state power, but the new state after a succession is not. Roughly speaking, 


If a city is absorbed via conquest, 


Past treatments of this topic have suffered from a lack of available historical data. To empirically tackle it one needs at least two things: the detailed territorial history of each city, and some city-level proxy for development, both reaching far enough into the past and ideally recorded in short intervals.


\cite{diamond1997} and \cite{landes2006} present narrative arguments for the economic benefits of state competition but stop short of formal analysis. 
More recently \cite{schoenholzer2022}, from whom I borrow extensively in this paper, use a dataset with a bigger geographical scope but lower resolution: they use urban population from \cite{bairoch1988} as a proxy for development, which is only available in intervals of 50 or 100 years. They are still able to estimate short-run (same-period) and long-run (next two centuries) effects of switching states. 
The dataset I use has several advantages. First, it traces the territorial history of cities in the Holy Roman Empire precisely up to a single year. Second, it contains 

Rather than m

\subsection*{Other literature}

\cite{diamond1997} asserts that the geographical remoteness of China deprived it of the benefits of state competition that European states enjoyed throughout their history. \cite{landes2006} makes a similar argument, pointing out how the totalitarianism and bureaucratism that stifled innovation in medieval China would not have survived long in a more fragmented state system.

\cite{levine2013, levine2021} provide the theoretical foundation for conflict-driven selection, as well as compelling evidence for it. 
They argue that social evolution is driven by conflict. Their model finds that "evolution selects societies with high state power - the ability to prevail in conflict over rivals." (\citealp{levine2021}, p. 436).Their answer to Why Europe and not China or India: "Ghengis Khan, the cannon, and the English. Ghengis Khan accomplished two things: he brought cannons to Europe, and he depopulated Mongolia. In China the removal of the outside threat of the Mongols led - as the theory asserts - to an extractive hegemony. In Europe the advent of the cannon together with the preying of the English on the French and the Spanish resulted - as the theory asserts - in an inclusive balance of power." (p. 439). strong outsiders but primitive siege technology [cheap defensibility of cities] in India led to extractive balance of power.  


How they define state power (pp. 439-440): "the potential ability of a social organisation to resist and influence other societies. This depends on institutional characteristics of the society, including stability of the government and the ability to collect taxes and conscript soldiers, which depend in turn on the law-abidingness of citizens, the efficiency of the courts, and the overall economic strength of the society." They *assume* in their model (p.443) that state power is directly proportional to free resources (resources in excess of subsistence).


I do not study their explanation for the European equilibrium that we observe in history; rather, I take a step back by taking the state system and the process of conflict-driven evolution as given. What interests me are the consequences of this process for economic growth.
Is it, as \cite{schoenholzer2022} argue explicitly, beneficial creative destruction - does state power equal state quality? Or are powerful states detrimental after all - because they engage in more extraction? \cite{tilly1985}


Why we would expect the hypothesis to be true: state power, defined as the ability to win conflicts with rivals, consists of factors like fiscal capacity, efficient courts, law-abiding citizens (\citealp{levine2021}, pp.439-449) - institutional factors that are also widely held to be beneficial for growth (see the vast literature on institutions). Add some institutions lit here.


My central identifying assumption: conquering a city implies more state power than inheriting it.






Key motivation papers: \cite{levine2013, levine2021}, \cite{diamond1997}, \cite{schoenholzer2022}, \cite{cervellati2022}, \cite{voigtlnder2013}, \cite{tilly1985}

The idea that today's states are just the most successful protection rackets of the past already shows up in \cite{tilly1985}.

While frequent wars brought devastation over the people and destruction to property, \cite{malthus1798} and others after him have argued, frequent wars may have kept population in check, leading to higher land-labour ratios and therefore a higher level of income per capita. \cite{voigtlnder2013} argue that the relative wealth of Europe after 1350 can be explained by a feedback loop centered on this Malthusian mechanism: after the huge loss of life caused by the Black Death, which served as a catalyst, taxable incomes rose, which made rulers richer, which allowed them to afford more wars, which in turn kept population in check.

% ######################################################

\section{Historical Background}

% ######################################################


% ######################################################



\newpage
\onehalfspacing
\bibliographystyle{chicago}
\bibliography{references.bib}

\newpage
\section*{Tables and figures}


\end{document}
