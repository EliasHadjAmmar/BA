%%%%%%%%%%%%%%%%%%%%%%%%%%%%%%%%%%%%%%%%%
% Beamer Presentation
% LaTeX Template
% Version 1.0 (10/11/12)
%
% This template has been downloaded from:
% http://www.LaTeXTemplates.com
%
% License:
% CC BY-NC-SA 3.0 (http://creativecommons.org/licenses/by-nc-sa/3.0/)
%
%%%%%%%%%%%%%%%%%%%%%%%%%%%%%%%%%%%%%%%%%

%----------------------------------------------------------------------------------------
%	PACKAGES AND THEMES
%----------------------------------------------------------------------------------------

\documentclass{beamer}

\mode<presentation> 
{

\usetheme{Madrid}

}
\usepackage{graphicx} % we need this so we can add figures
\usepackage{booktabs} % Allows the use of \toprule, \midrule and \bottomrule in tables
\usepackage{natbib} % we need this so we can use citation and bib properly
\usepackage{hyperref} % allows you to add hyperlink
\usepackage{amsmath} % allows you to use most mathematical features
\usepackage{setspace} % allows you to change the line spacing
\usepackage{longtable}
\usepackage{xr}
\usepackage{booktabs} % Allows the use of \toprule, \midrule and \bottomrule in tables
\usepackage{longtable}
\usepackage{threeparttable}
\usepackage{ragged2e}


\title[Kolloquium]{Conflict-Driven Evolution and Urban Development} % The short title appears at the bottom of every slide, the full title is only on the title page
\subtitle{Evidence from the Holy Roman Empire}

\author{Elias Hadj Ammar} % Your name

\institute[LMU] % Your institution as it will appear on the bottom of every slide, may be shorthand to save space
{
LMU Munich\\ % Your institution for the title page
\medskip
\textit{Elias.Ammar@campus.lmu.de} % Your email address
}
\date{August 3, 2023} % Date, can be changed to a custom date



\begin{document}

\begin{frame}
\titlepage % Print the title page as the first slide
\end{frame}

% no table of contents



%----------------------------------------------------------------------------------------
%	PRESENTATION SLIDES
%----------------------------------------------------------------------------------------


%------------------------------------------------
\section{Motivation} % 
%------------------------------------------------

\begin{frame}
\frametitle{Motivation}


\end{frame}


\begin{frame}
\frametitle{Motivation}
    
\end{frame}


%------------------------------------------------
\section{Outline} % 
%------------------------------------------------

\begin{frame}
\frametitle{Summary:}
\justifying
\begin{itemize}
\item \textbf{Research Question}: It should be a question, not a statement here. If you think the author has asked multiple questions, pick up to three important ones. 
\item \textbf{Main Methods}: Briefly mention their main identification strategy. What data did they use. 
\item \textbf{Main findings}: pick up to three findings. You can use key numbers here if you think they are important. Like this policy has reduced 10\% pollution.(Just an example)
\end{itemize}

\end{frame}

%------------------------------------------------


\begin{frame}
\frametitle{Positive View: }

\begin{block}{Point 3}
Here briefly mention one main positive view
\end{block}

if you want to add graphs, it is the same procedure as when you write your essay. Make sure you adjust the size so your classmates can see it properly.
\begin{figure}[hbt!]
    \centering
    \includegraphics[width=0.3\textwidth]{paper/output/descriptive/map_cities_sample.png}
    \caption{The composition of migrants and the provision of common schools; Source: me}.
    \label{table:my_label}
\end{figure}

\end{frame}

%------------------------------------------------

\begin{frame}
\frametitle{Positive View: }
You can also draw a table yourself

\begin{table}
\begin{tabular}{l l l}
\toprule
\textbf{Treatments} & \textbf{Response 1} & \textbf{Response 2}\\
\midrule
Treatment 1 & 0.0003262 & 0.562 \\
Treatment 2 & 0.0015681 & 0.910 \\
Treatment 3 & 0.0009271 & 0.296 \\
\bottomrule
\end{tabular}
\caption{Table caption}
\end{table}

\end{frame}



%------------------------------------------------
\section{End}% 
%------------------------------------------------

\begin{frame}
\frametitle{References}
\justifying

\bibliography{paper/references}
\bibliographystyle{chicago}

\end{frame}

%------------------------------------------------

\begin{frame}
\Huge{\centerline{The End}}
\end{frame}

%----------------------------------------------------------------------------------------

%------------------------------------------------
\section{Appendix}% 
%------------------------------------------------



\end{document}